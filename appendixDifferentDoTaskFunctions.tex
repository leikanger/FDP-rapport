
%TODO Sjekk også nederst.


%\section{s\_synapse}
%	When the \emph{s\_synapse} does its task, the postsynaptic potential is altered. 
%	As this does not happen instantaneously, it is important to let the action be carried out the next time iteration. 
%	The task done by each \emph{time\_interface} derived class' \emph{doTask()} function is listed in appendix \ref{appendixDifferentDoTaskFunctions}.

%\section{\emph{s\_auron}}
%\begin{lstlisting}
%\end{lstlisting}

%XXX dårlig tittel: finn bra tittel XXX
\chapter{Important Funcitons for the Workings of the Node} % TODO Her skal det som gjennomfoeres av de ulike elementas' doTask() listes opp. XXX Finn bra tittel. 
\label{appendixDifferentDoTaskFunctions} 

% XXX Denner kul, men kanskje ikkje den skal være med?
\begin{lstlisting}
/*************************************************************
****** 													******
****** 		  doTask() of important elements: 			******
****** 													******
*************************************************************/
\end{lstlisting}


\section{The Synapse: doTask()}
\label{appendixDoTaskForSynapse}
\subsection{\emph{s\_synapse::doTask()}}
\begin{lstlisting}
inline void s_synapse::doTask()
{
	// If synapse is exitatory : add w_ij to postsyn. value.
	// If synapse is inhibitory: subtract w_ij
	// Written by (1- 2*bInhibitoryEffect) dSynapticWeight  to 
	// 		make execution more effective.
 	pPostNodeDendrite->newInputSignal( (1-2*bInhibitoryEffect) 
									* 	( 1000 * dSynapticWeight)); 

	// Write syn. weight to log:
	synTransmission_logFile <<"\t" <<time_class::getTid() <<"\t" 
						<<(1-2*bInhibitoryEffect)*dSynapticWeight
						<<" ;   \t#Synpaptic weight\n" ;
	synTransmission_logFile.flush();
}
\end{lstlisting}




\subsection{\emph{K\_synapse::doTask()}}
\begin{lstlisting}
inline void K_synapse::doTask()
{ 
	// K_ij = w_ij / p(K_j).
	// Constant w_ij : 		Delta{K_ij} = w_ij / Delta{p(K_j)}
	pPostNodeDendrite->newInputSignal( 
			(1-2*bInhibitoryEffect) * dSynapticWeight  
			* (pPreNodeAxon->pElementOfAuron)->dChangeInPeriodINVERSE );

	// Write synaptic transmission to log.
	synTransmission_logFile
		<<time_class::getTid() <<"  \t"
		<<(1-2*bInhibitoryEffect) * dSynapticWeight  
		* (pPreNodeAxon->pElementOfAuron)->dChangeInPeriodINVERSE
		<<" ; \n";
	synTransmission_logFile.flush();
} //}
\end{lstlisting}



%----------------------------------------------------------------

\section{The Dendrite: doTask()}
\label{appendixDoTaskForDendrite}


\subsection{\emph{s\_dendrite::doTask()}}
\begin{lstlisting}
inline void s_dendrite::doTask()
{ 
	// Add the auron to pWorkTaskQue : time delay.
	time_class::addTaskIn_pWorkTaskQue( pElementOfAuron );
}

\end{lstlisting}

\subsection{\emph{K\_dendrite::doTask()}}
\begin{lstlisting}
inline void K_dendrite::doTask()
{
	// K_dendrite::doTask() is never to be used. 
	// Terminate progam and return error.
	exit(-1);
}

\end{lstlisting}


\section{Other functions important for the workings of the node}

\subsection{\emph{K\_dendrite::newInputSignal()}}
\label{appendixNewInputSignalDendrite}
\begin{lstlisting}
inline void s_dendrite::newInputSignal( double dNewSignal_arg )
{
	// Check if input is blocked, due to refraction period.
 	if( bBlockInput_refractionTime ) return;

	// Calculate leakage of value since last calculation:
	calculateLeakage();

	// Add new input to the value of the node.
	//   Scale input by time constant.
	pElementOfAuron->dAktivitetsVariabel +=  dNewSignal_arg * ALPHA;

	// Save the time of last input, 
	// 	for the use in calculatation of leakage.
	pElementOfAuron->ulTimestampLastInput = time_class::getTid();

	// If value goes over threshold: fire A.P.
	if( pElementOfAuron->dAktivitetsVariabel > FYRINGSTERSKEL )
	{
		// Set refraction time bool:
		bBlockInput_refractionTime = true;

		//Initiate action potential next iter. (time-delay):
		time_class::addTaskIn_pWorkTaskQue( pElementOfAuron );
	}

	// Write log for activity variable:
	pElementOfAuron->depol_logFile 	<<time_class::getTid() <<"\t" 
								<<pElementOfAuron->dAktivitetsVariabel 
								<<"; \t #Depolarization\n" ;
	pElementOfAuron->depol_logFile.flush();
}

\end{lstlisting}


\subsection{\emph{K\_dendrite::newInputSignal()}}
\begin{lstlisting}

inline void K_dendrite::newInputSignal( double dNewSignal_arg )
{
	pElementOfAuron->changeKappa_derivedArg( dNewSignal_arg );
}
\end{lstlisting}


%\subsection{---------------------------------------}
%-\\
%
%
%
%void time_class::doTask()
%{ 	//{ 
%
%	// gjennomfoerer planlagte kalkulasjoner:
%	doCalculation();
%
%
%	//itererer time:
%	ulTime++;
%
%	// utskrift:
%	#if DOT_ENTER_UTSKRIFT_AV_TID
%	cout<<".";
%	if(ulTime % DOT_ENTER_UTSKRIFT_AV_TID == 0)
%		cout<<endl;
%	#endif
%
%	#if UTSKRIFT_AV_TID
%	if(ulTime % UTSKRIFT_AV_TID_KVAR_Nte_ITER  == 0)		
%		cout<<"\t* * * * TID: \t  =  " <<ulTime <<" * * * * * * * * * * * * * * * * * * * * * * * * * * * * * * * * * * * * * * * * * * * * * * * * * * * * * * * * * * * * * * * * * * * * * * * * = "
%			<<ulTime <<"\n";
%	#endif
%
%
%	/*******************************
%	* Oppdater alle K_sensor_auron *
%	*******************************/
%	#if KANN
%	K_sensor_auron::updateAllSensorAurons();
%	#endif
%
%	/*******************************
%	* Oppdater alle K_sensor_auron *
%	*******************************/
%	#if SANN
%	s_sensor_auron::updateAllSensorAurons();
%	#endif
%
%	
%	/*************************************************
%	* Flytter planlagde oppgaver over i pWorkTaskQue *
%	*************************************************/
%	for( std::list<timeInterface*>::iterator pPE_iter = pPeriodicElements.begin() ; pPE_iter != pPeriodicElements.end() ; pPE_iter++ )
%	{
%		if( (*pPE_iter)->ulEstimatedTaskTime_for_object == ulTime+1 )
%		{
%			addTaskIn_pWorkTaskQue( (*pPE_iter) );
%			// Dette foerer til eit kall til eit tidsElements doTask(). Teller antall kall (til rapporten):
%			comparisonClass::ulNumberOfCallsTo_doTask ++;
%			#if DEBUG_UTSKRIFTS_NIVAA > 2
%			cout<<"Telte kall til " <<(*pPE_iter)->sClassName <<".doTask()\n";
%			#endif
%		}
% 	}
%
%	// Legger til egenpeiker paa slutt av pNesteJobb_ArbeidsKoe
%	pWorkTaskQue.push_back(this);	
%}//}
%
%
% //{
%\begin{lstlisting}
%
%/*
% //{   *  i_axon::doTask() KOMMENTERT UT
%inline void i_axon::doTask()
%{
% 	cout<<"i_axon::doTask()\tLegger inn alle outputsynapser i arbeidskoe. Mdl. av auron: " <<pElementOfAuron->sNavn <<" - - - - - - - - - - - - - - - \n";
%
%	// For meir noeyaktig simulering av tid kan alle synaper faa verdi for 'time lag' foer fyring. No fokuserer eg heller paa effektivitet. 
%		//			TODO gjoer om x++ til ++x, siden ++x slepper aa lage en "temporary".
% 	for( std::list<i_synapse*>::iterator iter = pUtSynapser.begin(); iter != pUtSynapser.end(); iter++ )
%	{ // Legger alle pUtSynapser inn i arbeidskoe: (FIFO-koe)
%		
%		//time_class::pWorkTaskQue.push_back( (*iter) );
%		// Legger til ut-synapser i time_class::arbeidskoe
%		time_class::addTaskIn_pWorkTaskQue( *iter );
%	}
%
%} //}
%*/
%//{ 		* 	SANN
%inline void s_auron::doTask()
%{ //{ ** AURON
%
%	//Avblokkerer bBlockInput_refractionTime: Kanskje dette skal vaere her? (1ms refraction time..)
%	//pInputDendrite ->bBlockInput_refractionTime = false;
%
%
%
%
%	#if DEBUG_UTSKRIFTS_NIVAA > 0
%	cout<<"\tS S " <<sNavn <<" | S | " <<sNavn <<" | S | S | S | | " <<sNavn <<" | S | | " <<sNavn <<" | S | | " <<sNavn <<"| S | S | S | S |\t"
%		<<sNavn <<".doTask()\tFYRER neuron " <<sNavn <<".\t\t| S S | \t  | S |  \t  | S |\t\tS | " <<time_class::getTid() <<" |\n"
%		<<endl;
%	#endif
%
%	//Axon hillock: send aksjonspotensial 	-- innkapsling gir at a xon skal ta haand om all output. // bestiller at a xon skal fyre NESTE tidsiterasjon. Simulerer tidsdelay i a xonet.
%	time_class::addTaskIn_pWorkTaskQue( pOutputAxon );
%
%
%	if( ulTimestampForrigeFyring == time_class::getTid() )
%	{
%		#if DEBUG_UTSKRIFTS_NIVAA > 1
%		cout<<"\n\n************************\nFeil?\nTo fyringer paa en iterasjon? \nFeilmelding au#103 @ auron.h\n************************\n\n";
%		#endif
%		return;
%	}
%
%	// Registrerer fyringstid (for feisjekk (over) osv.) 
%	ulTimestampForrigeFyring = time_class::getTid();
%
%
%	//Resetter depol.verdi 
%	dAktivitetsVariabel = 0; 
%
%	writeAPtoLog();
%
%} //}
%inline void s_axon::doTask()
%{ //{ ** AXON
%
%DEBUG("s_axon::doTask() START");
%	// Avblokkerer dendritt. Opner den for meir input. Foreloepig er dette maaten 'refraction time' funker paa.. (etter 2 ms-dendrite og auron overfoering..)
%	//static_cast<s_dendrite*>(pElementOfAuron->pInputDendrite)->bBlockInput_refractionTime = false;
%	pElementOfAuron->pInputDendrite ->bBlockInput_refractionTime = false;
%
%	#if DEBUG_UTSKRIFTS_NIVAA > 3
% 	cout<<"s_axon::doTask()\tLegger inn alle outputsynapser i arbeidskoe. Mdl. av auron: " <<pElementOfAuron->sNavn <<" - - - - - - - - - - - - - - - \n";
%	#endif
%
%	// Legger til alle utsynapser i pWorkTaskQue:
%		//			TODO gjoer om x++ til ++x, siden ++x slepper aa lage en "temporary".
% 	for( std::list<s_synapse*>::iterator iter = pUtSynapser.begin(); iter != pUtSynapser.end(); iter++ )
%	{ // Legger alle pUtSynapser inn i time_class::pWorkTaskQue: (FIFO-koe)
%		time_class::addTaskIn_pWorkTaskQue( *iter );
%	}
%
%	 //Skriver til logg etter refraction-period.
%	 pElementOfAuron->writeDepolToLog();
%DEBUG("s_axon::doTask() SLUTT");
%
%} //}
%//}            *       slutt SANN
%
%//{ 		* 	KANN
%
%
%inline void K_auron::doTask()
%{ //{ ** AURON
%
%	//{ Kommentert ut
%	// Feilsjekk! Sjekker om kappa har vore over Tau, siste 'time window'
%	#if 0
%	if( !bKappaLargerThanThreshold_lastIter )
%	{
%		#if DEBUG_UTSKRIFTS_NIVAA > 1
%		cout<<"ERROR \t ERROR \t ERROR \t ERROR \t Fyring mens K<T forrige time window: bKappaLargerThanThreshold_lastIter: false" <<endl;
%		#endif
%		return;
%	}
%	#endif
%	//}
%
%	#if DEBUG_UTSKRIFTS_NIVAA > 3
%	cout 	<<"K_auron::doTask():\n"
%			<<"\tdLastCalculatedPeriod :\t" <<dLastCalculatedPeriod <<endl
%			<<"\tdPeriodINVERSE :\t" <<dPeriodINVERSE <<endl;
%	#endif
%
%
%	//Utskrift til skjerm:
%	#if DEBUG_UTSKRIFTS_NIVAA > 0
%
%
%
%	cout<<"\tK K " <<sNavn <<" | K | " <<sNavn <<" | K | K | K | | " <<sNavn <<" | K | | " <<sNavn <<" | K | | " <<sNavn <<"| K | K | K | K |\t"
%		<<sNavn <<".doTask()\tFYRER neuron " <<sNavn <<".\t\t| K | |  [periode] = " <<dLastCalculatedPeriod/1000 <<"     | K | \tK | " <<time_class::getTid() <<" |\n"
%		<<"\t| | Kappa:" <<dAktivitetsVariabel <<"\tForrige periode:" <<dLastCalculatedPeriod <<"\tDepol foer fyring:" <<dDepolAtStartOfTimeWindow <<"\n"
%		<<endl;
%
%	#endif
%
%	//Axon hillock: send aksjonspotensial 	-- innkapsling gir at a xon skal ta haand om all output. // bestiller at a xon skal fyre NESTE tidsiterasjon. Simulerer tidsdelay i a xonet.
%	// TODO Dette kan gjoeres direkte, for aa slippe litt jobb: Dersom vi legger til 1 i fyringsestimatet, er ikkje axon-delay naudsynt.. TODO
%	time_class::addTaskIn_pWorkTaskQue( pOutputAxon );
%
%	// Setter v_0 til 0 og t_0 til [no]:
%%	dDepolAtStartOfTimeWindow = 0;
%	ulStartOfTimewindow = time_class::getTid();
%
%		// Har testa litt, og det er definitivt best aa kjoere neste tre linjene, iforhold til aa kjoere doCalculation()!
%		//(doCalculation() varianten slutta aldri paa 40K tidssteg, neste-tre-linje varianten slutta etter 30 sekund paa 1000K tidsiterasjoner..
%		//
%		// Har testa det for nye varianten av tidsplanlegging (sjekke alle aurons ulEstimatedTaskTime_for_object)
%		// Naa er det det samme (1 000 000 iter, to ukobla auron(eit vanlig og eit sensor) : 28,155 sek  eller 29.286  for doCalculation() VS 		29.056 eller  28.828 	for neste 4 linjene
%
%	//Har funnet ut at neste fire linjene er mest effektivt, men bare litt. For eit auron, 300000 tidsiter gjorde det 29/15/12 istedenfor doCalculation() som gjorde 31/17/13
%	#if 0
%	//{
%		// Beregn ny isi-periode^{-1}. Brukes til aa beregne syn.overfoering seinare i signal-cascade.
%		dL astCalculatedPeriod = (- log((dAktivitetsVariabel - FYRINGSTERSKEL) / dAktivitetsVariabel) / ALPHA);
%		
%		double dPeriodInverse_temp =  (1/d LastCalculatedPeriod);
% 		dChangeInPeriodINVERSE = dPeriodInverse_temp - dPeriodINVERSE ;
%		dPeriodINVERSE = dPeriodInverse_temp;
%	//}
%	#else
%		//Kjoer heller det over enn .doCalculations() ... Der er jo en del meir jobb.. (Krever sattans mykje meir arbeid!)
%		doCalculation();
%	#endif
%
%
%	// Oppdaterer ulEstimatedTaskTime_for_object til [no] + dLastCalculatedPeriod:
%	ulEstimatedTaskTime_for_object = time_class::getTid() + (unsigned long)dLastCalculatedPeriod;
%
%	// Logger AP (vertikal strek)
%	writeAPtoLog();
%	
%	
%} //}
%inline void K_axon::doTask()
%{ //{ ** AXON
%	#if DEBUG_UTSKRIFTS_NIVAA > 3
% 	cout<<"K_axon::doTask()\tLegger inn alle outputsynapser i arbeidskoe. Mdl. av auron: " <<pElementOfAuron->sNavn <<" - - - - - - - - - - - - - - - \n";
%	#endif
%
%	// Refraction time: Setter depol til 0 igjen (dette er 
%	// FUNKER IKKJE: (blir bare problemer med jaevvla hoege verdier (type folde-shit (eller ka-det-ne-neiter) )
%	//Ok, no funker det men har ingen effekt. (fjaerna +1 paa ulStartOfTimewindow=tid +1;
%	//pElementOfAuron->dDepolAtStartOfTimeWindow = 0;
%	//pElementOfAuron->ulStartOfTimewindow = time_class::getTid() ;
%
%	// Overfoering:
%	doTransmission();
%
%	// Gjoeres i s_auron. Skal eg ogsaa gjoere det her (for aa poengtere refraction time)?
%	//pElementOfAuron->writeDepolToLog();
%
%} //}
%
%// Delte opp for aa kunne skille mellom aa propagere kappa, og aa fyre AP, for aa kunne ha refraction period.
%inline void K_axon::doTransmission()
%{
%	// Legg til alle utsynapser i pWorkTaskQue:
%		//			TODO gjoer om x++ til ++x, siden ++x slepper aa lage en "temporary".
% 	for( std::list<K_synapse*>::iterator iter = pUtSynapser.begin(); iter != pUtSynapser.end(); iter++ )
%	{ // Legger alle pUtSynapser inn i time_class::pWorkTaskQue: (FIFO-koe)
%		time_class::addTaskIn_pWorkTaskQue( *iter );
%	}
%}
%
%
%//}            *       KANN slutt
%
%
%
%
%
%
%
%
%\end{lstlisting}
%
%
%
%
%
%
%
%
%
% //}
%
%%%%%%%%%%%%%%%%%%%%%%%%%%%%%%%%%%%%%%%%%%%%%%%%%%%%%%%%%%%%%%%%%%%%
