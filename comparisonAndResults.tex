
%TODO må skrives: sett av 2 dager

% Sammenligning:
% Oppgave 3) Compare the new concept with competing methods. The evaluation may be based on a number of specific scenarios (with respect to input data etc.).
% 	Skriv at konkurerende metoder må tolkes til SANN, siden det blir for mykje å sammenligne med fANN også. Dette er en av styrkene til KANN: Kan snakke med alle!
% 	Skriv at for å kunne sammenligne med SANN, så har eg implementert begge. Dette gjør at eg kan meir grundig sammenligne implementering av de to. 
% 		Eg blir også bedre kjendt med implementasjonen så lettere å seinere gjennomføre en vilkensomhelst test (ikkje med rammer som er satt av implementasjonen).
%
% 	- Sammenligning om implementasjon OG kjøring om: Bra/dårlig ved:
% 		- kjøring av de to
% 		- implementasjon av de to
% 		- sensor(best for KANN)
% 		- kjøretid (effektivitet for ulike 'scenarios') - DISKURS
% 		- over nettverk : best for KANN. 				- DISKURS
%


% TODO TODO T0D0:
% Verifiser all koden som står i teksten.



%Intro til kapittelet:
%	Intro: oppsummere kva som er gjort i dette prosjektet:
%		- Modellere neuronet.
%		- Utvikle KANN (som resultat av ligninene fra over)
%		- Utvikle SANN. Laga selv for best å kunne sammenligne forskjellene.
%		For begge ANN:  Sterk inspirert av biologien.
%			-> i tillegg til fordelane som er nevnt tidligare: gir eit felles design for implementasjonane av de to modellane => Meir sammenlignbar.
In this project two artificial neural networks have been designed and implemented. 
For the implementation it has been focused on comparability, both in respect to the design of the simulation and the behavior of the indivitual node.
The systems will be more comparable if both systems have the same blueprint. 
This is one of the reasons why the implementations are so inspired by biology.
An other cause for the strong resemblance with the biological neuron is the recognised connection between each generation of ANN; As ANN models evolve, each generation become closer to the biological version. %skriv om xxx.
This observation is a strong motivation of make the impementation close to the biological neuron, as long as it does not affect the effectivity of the simulator. %bruk fleire setninger?
%The comparability in design and implementation it the cause of an, in some cases, overly complicated object model.

For comparing the behavior of the indivitual nodes of the two nodels, some effort was put into a capabel logging system. 
Different elements of the node with interresting variables of effects were logged by this system. This includes the \emph{K\_synapse}, the \emph{K\_auron} and the \emph{s\_auron}.
The synapse was logged mainly for design purposes. The two auron elements were logged for design purposes and for the ``depolarization'' value comparison.
Both auron models is able to write the ``depolarization'' to a log file. The $\kappa$N is also able to log it's activity level, through $\kappa$.

The activity level of a spiking neuron us not formalized; Neuroscientist often use variables such as the firing frequency to measure the activity of a neuron.
For this reason, no method has been implemented for logging the activity level of the SANN nodes.
The activity of the single node is a phenomenon that first becomes important when we analyze larger neural networks.
For the comparison of the single node, the value of the node is the best comparison variable. 
%This containt more information, and if analyzed also contains information of the firing frequency and the synaptic input of the node could be found.
In this report it is most interresting with the transient depolarization curve of the nodes.




\section{Design and Implementation of the Two Models}

	% Begynn med å oppsummere korleis objektmodellen er lagt opp.
	% Gå gjennom de ulike elemena? Auron -> dendrite -> osv. ? 	 (HUSK Dette er bare innledning: skal være en ekte delmengde av det som kommer, (delmengde som ikkje er det samme), DERMED: IKKJE SKRIV ALT ("hold spenninga oppe").
	% (Skriv om ulikhetane i variabler. Bare såvidt nevn ulikhetane i funksjoner..)
	% 	[Slik:] For the auron, not much could be put into the common interface class of i_auron ...
	% 			For the axon, on the other hand ...
	% ANDRE forskjeller: (designforskjeller)
	% Design av de to implementasjonane er gjordt som en direkte simulering av neuronet (spiking neurons), med dendrite, axon, osv. 
	% 	Dette førte til at eg begynte å impelmentere de to som like, bare med peiker-funksjoner for de funksjonane som trenkte å være ulike.
	% 	Eg oppdaga raskt at dette ikkje ville fungere, siden nesten alt er ulikt fra SANN til KANN.
	% 	Endte opp med å ha en interface class for kvart element (i_[element]) som arva til de to modellenes design av elementet (s_elemen og K_element).
	% 		Dette førte til at forskjellane i design ble veldig tydlige. Fellestrekka for de to modellane ble også veldig tydlige.
	% nevne det som står kommentert ut i section{Design for each node, ...} ?
	%
	% Nevn også at objektmodellen la opp designet (for begge modellimplementasjoner), og eg oppfatta først etterpå at dette kunne gjøres langt meir effektivt for KANN.
	%  

	Skriv at objekt modellen var spesiallaga for å sammenligne de to. Det kom fram at for KANN er det mulig å gjennomføre tidsdelay som kommer av axon og dendrite med bare auron-elementet.
	Dette gjøres med å plusse på en på estimert fyringstid.. VIKTIG poeng!
	\subsection{Similarities and Differences of the indivitual node elements between the two models}
		\subsubsection{auron}
		\subsubsection{axon}
		\subsubsection{dendrite}
		\subsubsection{synapse}
	\subsection{Design for each node, when designed as a simulation of temporal aspects of the neuron}
	% XXX Dette skal bare nevnes her (eller?), men skrives så utførende som teksten under, i egen section.
	% En annen ting en oppfatta var at simulering av kvart element ikkje var nødvendig for simulering av time--delay for propagering av AP: for KANN kan dette inkorporeres i estimering av fyringstid og periode (=>syn.transmitted var.)
	% 	For KANN kan man lett effektivisere bort simulering av tidsdelay, ved å bare legge til tidsdelayet i estimat av fyretid. Dette er vanskeligare å gjøre i SANN. I KANN er denne typen effektivisering "native".
	% 	Dette gir at for simulerings-likheten som er brukt i dette systemet (en dendrite, eit axon -- alle synapser samme plass..), så er det mulig med eit--elements noder for KANN uten å miste simuleringslikhet.
	% 		I implementasjonen er K_dendrite::doTask() og K_dendrite::doCalculations() definert til å gi sterk melding og avslutte programmet. Dette for å eksemplifisere at de ikkje er i bruk, og forsikre at de ikkje blir kalla.
	% 		For K_axon så gjelder fortsatt det at eg vil ha muligheten til å auke 'spatsio-temporal' oppløsning. Dette gjøres ved å legge inn fleire element av K_axon. K_axon blir dermed som eit navier--stokes kontrollvolum for neuronet.
	% 		Dersom en pragmatisk implementasjon blir laget fra denne koden, kan dermed K_axon fjærnes, of listen med output synapser og doTransmission() kan legges over i K_auron.
	% 		I såfall får vi eit enklere oppsett of kvar node, med bare eit node element (K_auron) og eit 'edge' element (K_synapse), uten å miste kvalitet i simuleringen. 
	% 		Dette kan sees i plotta som sammenligner depol-kurve for de to ANN (sjå plott[_ref_])
	
	Konkluder med å diskutere lettheten i implementasjon(at KANN er vanskeligare å implementere, men med andre pluss..) 			ELLER:   i 'discussion and conclusion' ?

\section{Comparison of the nodes' output}
	% XXX Dette skal hovedsaklig ligge i KANN, men siste plott skal også legges ved her (eller bare her?) \subsec.{Comparison between the transient time course of the depolarization of the K auron and s auron,Vettafaen kor det skal ligge}
	\subsection{Comparison of the transient time course of nodes of the two models}
		Intro: Siden skifte av 'time window' fører til at eg regner ut depol. for KANN er det mulig å sammenligne de transiente kurvene til de to modellane.
		%TODO Flytt siste plottet over her, og konkluder først her at det var avrundingsfeil som gjorde avviket. Skriv også at analysen virker correct (om trunctation error).
		%[Resten er skrevet..], men må skrives om. Litt skal flyttes hit, slik at ivertfall siste plottet er her. Skrive at dette ble endra i implementasjonen, og vi fikk : SISTE PLOTTET.
	\subsection{Analysis of Synaptic Transmission as the derived}
		% Her skal plottet som gir det mest komplekse responsen ligge (ulineariteten og pausa).
	\subsection{Analysis of the postsynaptic activity variable as a result of syn. transmission} % KANSKJE
	\subsection{etc.}
\section{etc.}
