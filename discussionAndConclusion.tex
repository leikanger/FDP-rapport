
%TODO Skriv: Sett av 2 dager.



%todo todo todo todo todo todo todo todo todo todo todo todo todo todo todo todo todo todo todo todo todo todo todo todo todo todo todo todo todo todo todo todo todo 
% 	ENDRE TITTEL på forrige kap  til "comparison and discussion".      "comparison and discussion"       "comparison and discussion"       "comparison and discussion"       "comparison and discussion"       "comparison and discussion" 
% 		-> og endre tittel på dette kapittel til "Conclution"..
%todo todo todo todo todo todo todo todo todo todo todo todo todo todo todo todo todo todo todo todo todo todo todo todo todo todo todo todo todo todo todo todo todo 


a







% Skriv at:
% 	- 1. gen. ANN er stateless, men overfører boolst signal. (det er i tidsdomenet)
%  	- 2. gen. ANN er stateless, men i frekvensdomenet. Dette er rettere.
% 	- 3. gen. ANN er med state. Dette er også i tidsdomenet, så får med temporale effekter.
% TENKING: KANN er med med state, og tidsdomenet. 
% 	For KANN varierer output med input også, og kan sees på som en mealey variant av SANN. 
% 	Dersom vi definerer 3.gen. ANN som eit ANN der output ikkje varierer med input, vil ikkje KANN tilhøre 3.gen. ANN.
% 		- Det er mulig å hente ut informasjon om tid, dersom ønskelig (til f.eks. syn.p.), men ikkje naudsynt.
% 		- Trenger ikkje beregne neuralnettet som eit SANN. Dette har vore gjordt i denne rapporten, men trengs ikkje.
% 		- Skrive i denne sammenhengen (kanskje i resultat?) at i løpet av denne oppgaven har hovedfokus vore på sammenligninig. 
% 		- Siden modellen ikkje var ferdig før prosjektet, gjorde dette at implementasjonen ble spesialisert for sammenligning (brukte spiking variantav synaptisk transmissjon). Dette er antaglig ikkje optimalt.
%Kanskje lage tabell, eller begin{itemize} her, under:



%XXXXXXXXXXXXXXXX XXX XXXXXXXXXXXXXXXXXXXXXXX XXX
%skriv dette en eller anna plass, på en eller anna måte:
	%\subsection{Grunn til å gå vidare fra SANN til $\kappa$ANN. Ide for $\kappa$ANN}
	%	Kanskje eg skal her skrive om kva eg  reagerte på med SANN? (ikkje vent å sjå på resultatet, skriv uansett. Tolk heller at dette var feilt..)
%XXXXXXXXXXXXXXXX XXX XXXXXXXXXXXXXXXXXXXXXXX XXX







% Hugs modellen om syn.trans.  	Skriv i conclusion that a better model can be included with a greater ease, in the K-synaptic transmission than the s-synaptic t.
% 	(if I manage to modell such behaviour..)


% Diskurs:
% 	- diskuter tid: de fleste neuron med realistisk mengde input vil ha en input per tidsiter. Dette gir at de antagligvis MAKS får eit input per andre iter (dersom alle fyrer maksrate..)
% 		Også da vil det bli meir effektiv modell for tid, men virkelige nytten kommer nok med KANN: da kan vi nok ha 'abituary size' på tidssteget (vil ikkje innføre meir belastning (FOR STATISKE NETT(nett med statisk aktivitetsnivå)) )
% 		JEJE..
%
% 	- avviket mellom SN og KN etter synapse når synapse var skjempestor   (nei: dette er resultat og comparison= 
% 
% 

% Conclution
% 	- i dette prosjektet har neuronet blitt simulert. To simulatore er skrevet. For å gjøre dette, har eg lest meg opp på det neurale systemet.
%


% xxx skrive at det er bra med tidsinvariant overføring: Bl.a. kappa kan når som helst bli rekalkulert. Dersom ligninger for det blir utvikla, kan også kanskje v_0 bli rekalulert?

% XXX CONCLUSION : Eg har funnet at det er veldig bra å bruke Kappa som mål på aktivitetsnivå til neuronet. Dette kan være nyttig for neuroscience..

% XXX CONKLUSJON : skrive at modene i SANN (etablerte modellen) er en Moore variant av det biologiske neuronet. KANN er en Mealy automata av neuronet, noke som gir andre muligheter. Drøft.

% XXX CONKLUSJON : For fANN brukes gjennomsnittsfrekvens over time step. For KANN brukes estimert frekvens for å gi syn. trans.
% 					I denne oppgaven var det ikkje fokus, men dersom en bedre oppførsel ønskes, kan vi la estimerting av frekvens (for syn.trans.) være basert på metoder lært i kyb.
% 					Høgere ordens estimasjon vil nok heilt sikkert gi bedre oppførsel. No er overføringa basert på en naiv estimering av fyringsfrekvens.

% XXX SKRIV OM planning of other tasks: (Teksten er med i resultat-og-diskurs: "comparison of the individual elements" : "The mechnisms of the auron element")
% 	As this implementation have been based on the SANN model, the \emph{K\_axon} also makes use of the \emph{doTask()} function.
%	The alternative way of planning synaptic transmission and other planned tasks first became clear to the author durin the writing of this report, these aspects have not been implemented.
%	As the principle of task planning has been used for the auron element, and can easily be extended to be used for other elements, this is included in the discussion.
%	%The reasoning is sound, and the concept of estimated task time can easily be extended to synapses and other elements. Due to time constraints, this is unfortunately not implemented.


% 	xxx Skriv grunnen til at eg ikkje har nokon uventede resultater: at dette i all hovedsak har vert en utviklingsprosess, og alle uventede resultater har blitt ordnet på undervegs.
% 		Eksemper på slike er den lange diskusjonen tidligere, om avviket mellom K_auron og s_auron. Dette ble fikset. Mye tid har dermed blitt brukt på dette prosjektet.

% 	TODO TODO Poengter at synaptic transmission matematikk er basert på 2.gen. ANN (mens mykje anna er basert på 3.g). Dette kan diskuteres frem og tilbake!


%TODO : Skriv at selv om vi ikkje har fått tid til å kjøre effektivitetssammenligning, så har vi funnet eit viktig resultat. KN kan brukes til å oversette mellom 2.gen og 3.gen ANN.
% 		TODO Finn ut korleis 1.gen funker, og skriv evt. inn dette også. (sjå % ref_123@i_KANN i implementasjon_KANN.tex) EVT ikkje nevn "each of the other generation ANN" og "first gen. ANN" fra linja "% ref_123@i_KANN" står på.

% Kva er nytt i denne oppgaven:
% 	- time planning: kanskje KANN modellen er mindre effektiv, men elementet med time planning kan brukes for SANN også. Da slepper vi å simulere time delay.
% 		Axon blir dermed unødvendige, og vi kan lage eit scheme som bruker tidsplanlegging for forskjellige tasks. FUNKER FOR BEGGE MODELLER!
%
%


% Videre arbeid:
% 	- Lage syn. p.
% 	- Innføre dynamisk teskel (bedre modell for refraction time)
% 	- Foreta effektivitetsanalyse.
% 	- Lage transduction mellom KN og SN. Også mellom fN og KN. Kanskje heile veien, mellom alle generasjoner ANN.
% 	- Utvikle KANN for seg selv. Slik at dette blir så effektivt som mulig..
% 	- single compartment implemention of multiple compartment model of the neuron. (KANN, time scheduling, planlegging av tasks)
% 	- effektivitetsanalyse av dette.
%
%



% Oppsummering (conclution):
% 	innlede med kva som er  gjordt i dette prosjektet (gå gjennom kapittela og skriv ned..)
% 	- Skriv om transduction mellom ANN av ulike generasjoner. Kan bruke KANN til dette! (sjå sec \subsection{The Activity Variable} ).
% 	- Skriv at eg har gjordt den tidsvariante responsen til en overføring (at overføring skjer/ikkje skjer er avhengig av verdi før overføring. Dette skaper eit system som er avhengig av input-historie og tid(lekkasje).
% 		Om til: å ikkje lenger være tidsvariant. Kappa er på en måte tatt utafor tid. Kappa som overført variabel er ikkje tidsvariant, ikkje variant med andre variabler.
% 		=> Den tar vekk ulineariteten innført fra mekanismen: fyring!
%


% Feil/Mangler i denne sammenligninga:
% 	- Har ikkje sett på nettverkseffekta av tidspunkt for propagering av Kappa. Kan bli veldig rart at den propagerer "med en gang" (etter 'current' time iteration).
% 		%TODO analyser dette i diskurs!
% 	- Burde kanskje brukt meir tid på rapporten.
%
%
%
%
%

%%
%%XXXXXXXXXXXXXXXXXXXXXXXXXXXXXXXXXXXXXXXXXXXXXXXXXXXXXXXXXXXXXXXXXXXXXXXXXXXXXXXXXXXXXXXXX
%Skrive at design av/teorien bak  de to impelmentasjonene er så forsjellig at det er vanskelig å sammenligne de to med testing. En enkel kjøring vil ha statisk input (ikkje-endrende input).
%Mealy varianten av SANN (KANN) er spesialisert for ANN med dynamisk (endrende) input. Vil gi eit vanskeligere testoppsett for sammenligning av de to. Lett å implementere for KANN, vanskeligere å implementere for SANN.
%(Da må eg ha egene sensor-neuron som er spesiallaga for å sense en slik dynamisk state).
%%XXXXXXXXXXXXXXXXXXXXXXXXXXXXXXXXXXXXXXXXXXXXXXXXXXXXXXXXXXXXXXXXXXXXXXXXXXXXXXXXXXXXXXXXX
%



% XXX Ligger som conclusion:
		\subsection{The Differences between the Models} %From a broader perspective / A larger view / Discussion : /  				XXX Skriv en av desse først..
		From the previous section, the difference between the indivitual elements was discussed. 
		The elements was discussed one at a time, following the propagation of the signal in the biological neuron.

		In this section, the results from the comparison will be discussed with a broader view.
		The differences in general design and implementation will be discussed first.
		We will also discuss what effect this might have on the efficiancy of the execution for each node.


% KANSKJE DET HELLER ER CONCLUSION?
%	\subsection{Discussion about the differences in the Elements} %TODO Endre tittel!
		As the SANN model is a direct simulation of each node's depolarization, SANN requires the design used in this project.
		For simulation of SANN, the design of each node should therefore be as introduced first in section \ref{secTheBiologicalNeuralSystem} about the biological neural system, 
			and later in sec. \ref{secDesignForANN} about the design of this implementation.
		In SANN, ach node executes the same actions as the biological equivalent.
		As the SANN model does not make use of planned actions, time delay is implemented as a direct result of tranmission through elements with a time delay of one iteration.
		Time delays spanning multiple time iterations can be implemented by having multiple elements in series.

		%The alternative to implementing the  xxx skrive om single compartment model DERSOM TID.

		In $\kappa$ANN, the firing time of each node is estimated by the present level of input. 
		When the time estimate is the same as the next time iteration, the element is inserted into \emph{pWorkTaskQue}.
		%As the $\kappa$N does not have to transmit only as a consequence of the firing time, the calculation of firing time could be limited to the nodes in use of the firing time of the node. 
		% % When synaptic plasticity is implemented, this for example includes every node with a simulated glutamatergic transmission (as the output synapses of these nodes will have STDP)
		
		In the $\kappa$N, time planning it therefore the mechanism behind the firing of a node.
		Delays caused by simulated intracellular effects could therefore be implemented by simply adding the time delay to the estimated task time.
		This is an effect that could be implemented for the individual synapse;
			When the node fires, all its output synapses' \emph{ulEstimatedTaskTime} variable is written over with the value [now]+[corresponding delay].
		This gives that the $\kappa$N could make use of a single compartment implementation without using a single compartment model of the neuron.

\subsection{TEMATISKE FORSKJELLER : skaper muligheter i forhold til effektivitet!}
		As this implementation have been based on the SANN model, the \emph{K\_axon} also makes use of the \emph{doTask()} function.
		In this implementation, the \emph{K\_auron::doTask()} only calls the memberfunction \emph{doTransmission()}. This could as easily be called by a central element, or not at all. %For synapsene kan legges direkte inn som planl. oppg
		The alternative way of planning synaptic transmission and other planned tasks first became clear to the author durin the writing of this report, these aspects have not been implemented.
		As the principle of task planning has been used for the auron element, and can easily be extended to be used for other elements, this is included in the discussion.
		%The reasoning is sound, and the concept of estimated task time can easily be extended to synapses and other elements. Due to time constraints, this is unfortunately not implemented.


%fra auron: (tatt vekk derifra..)
		If the $\kappa$ANN is implemented by the 

%		As we could use the concept of task scheduling for different elements, simulation of intracellular propagation delay is unnecessary. %TODO TODO Skriv sammen med neste setning.
%		As the concept a simulated delay is obsolete in $\kappa$ANN, we do not have to propagate the signal through the node.
																			% it is unnecessary to simulate the signal propagation throught the internal elements of the node.
		% Vi kan dermed bare endre ulEstimatedTaskTime_for_object til ut-syn. ved fyring!
		If the $\kappa$ANN is implemented by following this realization, we could use a one--compartment implementation for a multiple--compartment model of the neuron.
		%Among other improvements, the execution of the $\kappa$N could be done without the use of dereferencing pointers when variables are to be accessed.

		In this case, the $\kappa$N is the only internal element of each node. The synapse is then percieved as the edge between two nodes in the graph.
		The $\kappa$N is therefore responsible for estimating the task time of every element of the node and its output synapses.
%TODO TODO TODO TODO TODO TODO TODO TODO TODO TODO TODO TODO TODO TODO TODO TODO TODO TODO TODO TODO TODO TODO TODO TODO TODO TODO TODO TODO TODO TODO TODO TODO TODO TODO TODO TODO TODO TODO TODO TODO TODO TODO TODO TODO 
		It is therefore responsen
		The $\kappa$N is then the only element in each node of the ANN, and is responcible to estimate task time for the auron and the output synapses.
		As this implementation used the SANN model as a basis, time delay is simulated by executing serial linked elements.
		This is unnecessary for $\kappa$ANN as it has the capability of time scheduling of tasks.

%		Examples of tasks that are to be executed after firing of the node is calculation of STDP of the node's synapses and transmission throught synapses going to output nodes of the SANN model.
%		Neither of these aspects have not been implemented in this project, but the mechanisms used by the $\kappa$N is easily implemented.




%% XXX TEMATISKE FORSKJELLER -- Når har vi høg load for systemet. Når kreves mykje ressurser?
%	\subsection{Høgare ordens mekanismer som resultat av det over. Forskjeller..} 
%	%Det kom fram at for KANN er det mulig å gjennomføre tidsdelay som kommer av axon og dendrite med bare auron-elementet.
%	%Dette gjøres med å plusse på en på estimert fyringstid.. VIKTIG poeng!
%	%
%	% XXX Skriv om reactive vs. proactive kalkulering av fyringstid.
%	\subsection{Design for each node, when designed as a simulation of temporal aspects of the neuron}
%	% XXX Dette skal bare nevnes her (eller?), men skrives så utførende som teksten under, i egen section.
%	% En annen ting en oppfatta var at simulering av kvart element ikkje var nødvendig for simulering av time--delay for propagering av AP: for KANN kan dette inkorporeres i estimering av fyringstid og periode (=>syn.transmitted var.)
%	% 	For KANN kan man lett effektivisere bort simulering av tidsdelay, ved å bare legge til tidsdelayet i estimat av fyretid. Dette er vanskeligare å gjøre i SANN. I KANN er denne typen effektivisering "native".
%	% 	Dette gir at for simulerings-likheten som er brukt i dette systemet (en dendrite, eit axon -- alle synapser samme plass..), så er det mulig med eit--elements noder for KANN uten å miste simuleringslikhet.
%	% 		I implementasjonen er K_dendrite::doTask() og K_dendrite::doCalculations() definert til å gi sterk melding og avslutte programmet. Dette for å eksemplifisere at de ikkje er i bruk, og forsikre at de ikkje blir kalla.
%	% 		For K_axon så gjelder fortsatt det at eg vil ha muligheten til å auke 'spatsio-temporal' oppløsning. Dette gjøres ved å legge inn fleire element av K_axon. K_axon blir dermed som eit navier--stokes kontrollvolum for neuronet.
%	% 		Dersom en pragmatisk implementasjon blir laget fra denne koden, kan dermed K_axon fjærnes, of listen med output synapser og doTransmission() kan legges over i K_auron.
%	% 		I såfall får vi eit enklere oppsett of kvar node, med bare eit node element (K_auron) og eit 'edge' element (K_synapse), uten å miste kvalitet i simuleringen. 
%	% 		Dette kan sees i plotta som sammenligner depol-kurve for de to ANN (sjå plott[_ref_])
%	
%	Konkluder med å diskutere lettheten i implementasjon(at KANN er vanskeligare å implementere, men med andre pluss..) 			ELLER:   i 'discussion and conclusion' ?

%%\section{Forskjell i implementasjon}
%Skrive at forskjellen mellom gamle og nye varianten av 3.gen. ANN (SANN) har mykje til felles med forskjellane mellom Moore vs. Mealy Auromata.
%
%Den gamle varianten er for så vidt den som er mest intuitiv å implementere / designe. Denne ser på umiddelbar tilstand (depol. verdi) for nodene. Dersom denne depol. kommer over terskel vil noden gi output til alle sine utnoder.
%Dette er en direkte simulering av det biologiske neuronet. Kan beskrives (direkte?) med Moore Automata.
%
%For den foreslåtte varianten vil Mealy automata beskrive systemet bedre. Her er det 'state' for noden og i tillegg input som gir ut-oppførselen.
%[skriv kva "utoppførsel" betyr. Alltid samme output (før synapsen) for den Moore-Automata varianten av SANN. For Mealy variant vil output være en flyttallsvariabel]
%[skriv kvifor denne tanken kom -- at enkel kraftig prosessor vil være oppmot like effektiv med større flyttalsoperasjoner som ved enkle boolske transmissjoner (tjaneei..)
%Dersom vi har mulighet er ferre større operasjoner bedre for den serielle CPU enn mange små operasjoner.]
%
%[Skriv at implementasjonen viste seg å bli meir omfattende enn først tenkt. Skriv om pEstimatedTaskTime og anna ekstra tidsplanlegging]
%
%% TODO TODO TODO TODO TODO TODO TODO TODO TODO TODO TODO TODO TODO TODO TODO TODO TODO TODO TODO TODO TODO TODO 









































%\chapter{FRA GAMMEL comparison and results -- fil:}


Skrive innledning: Kva er felles for de to implementasjonane: Kva er kjendt i biologien, og kva er utelatt for desse implementasjonene?

Kva har dette til felles med andre implementasjoner, og kva er gjort bedre i denne impelementasjonen i forhold til andre varianter av ANN (1. gen., 2. gen. og 3.gen. ANN).








%XXX Skriv i conclusion:
%Because a network of connected neurons have a large degree of complexity, it is best to start with comparing the depolarization of single nodes.
%For comparison of the two models, we will first compare the depolarization of a single neuron of each model.
%Because a system comprised of a network of neurons have 

%IKKJE TA MED. Er ikkje heilt sikker på kvifor det er slik. Dersom eg skal ha det med, kan eg jobbe med dette når rapporten er ferdig!
%\section{Synaptic Transmission}
%Få heilt klar for deg koleis sammenehenen er mellom K syn.t. og s syn.t. er før dette skrives.
%
%When the comparison between the postsynaptisk depolarization began, it could seem like the equations for synaptic trantmission were wrong. 
%The postsynaptic node of a s\_synapse had a larger value that that of the postsynaptic $\kappa$N in a corresponding $\kappa$ANN circuit.
%After analyzing the result, it was found that the error followed the same principles that were discussed in section \ref{ssecTruncationErrorOfSN}.
%
%The initial testing of the different ANN models use few nodes with a strong connection between them.
%The synaptic connections had a synaptic weight that could make the postsynaptic node go to suprathreshold levels in less then ten transmissions.
%In biological networks and in useful artificial networks, a connection of this size would make the system %XXX XXX DÅRLIG (skriv dette på bedre måte..)
%
%The result of this was that the initial transmission of a SN would cause a jump to this level, and the 
% % TODO dersom eg skal skrive denne secion, så må eg legge ved plot:  ../FDP_ANN/datafiles_for_evaluation/sammenligningEtterSynapseMedKonstantPresynSensorFunk.eps !
%When the circuits were initially





\section{Discussion and Concluding Remarks}
In the course of this project what is thought to be a new model for artificial neural networks have been developed. 
Because the primary focus of the project was to compare the developed model for ANN with other ANNs with the ability to represent both firing time and firing frequency, the development and implementation recieved this focus. 
%The new model was therefore initially implemented as a mere variant of a third generation ANN, with respect to information propatation.
Because such a model could not be found, $\kappa$ANN was compared with the model capable of representing firing time, SANN.

The new model was initially implemented as similar to SANN as possible, to make the two models compatible.
This caused the initial implementation of $\kappa$ANN to be a mere variand of SANN, with respect to information propagation;
	The activity of the node propagated when the calculated value of the node went to suprathreshold levels.
%The new model was therefore initially implemented as a mere variant of a third generation ANN, with respect to information propatation.
%Initially, the activity of the nodes propagated when the calculated value of the node went to suprathreshold levels.
%When the model was futher developed, it became clear that  %TODO Skriv når, og kvifor(at da) kappa propagerer.

%If we define the third generation ANN as a network of nodes capable of calculating a state of the node based on past and present input, and by this state deside the firing time of the node, it is possible to places $\kappa$ANN in this group.
%If we define the third generation ANN as a network of nodes capable of integrating the input to a state for the node, and this state gives the firing time of the node, it is possible to place $\kappa$ANN in this group.
%If we define the third generation ANN as a network of nodes capable calculating the state (``depolarization'') for the node, and by this state deside the firing time of the node, it is possible to place $\kappa$ANN in this group.
If we define the third generation ANN as a network of nodes capable of firing a spike as the result of past and present input, $\kappa$ANN could fall under this category.
%If the output of a node is given as a spike following a history that cause the value of the node to go to suprathreshold levels, the implementation would fall under the previous definition of SANN.
An implementation where the output of a node is given as a spike following an imput history that cause the value of the node to go to suprathreshold levels, the implementation would fall under the previous definition of SANN.
%If we define the output of each node in $\kappa$ANN to be the spikes following a history that cause the value of the node to go to suprathreshold levels, the network would fall under the SANN catherory as previously defined.
The nodes of this implementation of $\kappa$ANN could be said to be a Mealey automata of the LIF model of the neuron, and could therefore give output to a SN. %SANN model. %BLI HEILT SIKKER PÅ MEALY AUTOMATA! TODO
%This is how it is implemented in this project.

%TODO Enten skriv koleis den propagerer, eller kvifor det faller utafor:
If the third generation ANN is defined as a network of nodes where the activity is propagated by spikes, $\kappa$ANN falls outside this category.
% 																									BEDRE:		input is integrated to ...
%If the third generation ANN is defined as a network of nodes where the activity is propagated with spikes, and the input of a node is integrated to become the value of the node, $\kappa$ANN falls outside this category.
%																											%and a node's value is the integral of its input, $\kappa$ANN falls outside this category.
In all versions of SANN the author has been able to find, synaptic transmission is based on adding the synaptic weight to the postsynaptic node's value. %the input of a node is integrated up to the value of the node.
The value of the node ``leaks'', or looses value as a funciton of time and the value of the node.
If the value goes to suprathreshold levels, the node sends a simulated action potential causing transmission in all its output synapses.
This could be said to be a direct simulation of a LIF neuron.
%This is in other words, quite different than the mechanisms of $\kappa$ANN nodes.

In $\kappa$ANN we have ``firing'' for a different reason, one is to be able to compare the transient time course of the value of the node.
%As firing of a node causes the value to be reset to zero, this is an important part of the inter--spike period for the neuron.
%To be able to simulate this and thus the spike time
%TODO  HER  TODO  HER  TODO  HER  TODO  HER  TODO  HER  TODO  HER  TODO  HER  TODO  HER  TODO  HER  TODO  HER  TODO  HER  TODO
This is done to be able to use mechanisms of learning called STDP, and to be able to communicate with SNs.
If a $\kappa$ANN implementation wants to communicate with a SN, this could be done by defining a special set of transductor synapses.
These synapses would use the synaptic transmission scheme from the SANN model, and ouput would only be sent through it after a spike(after the right time delay).
%sending output through it after a spike (after the right time delay).


% 		% TODO Få med : Differs FROM SANN 				     XXX Sjekk det som kommer etter forrige "XXX" (Skrevet litt for seint..)
In this implementation, the time of propagation also differs from that of the SANN model. 
It was found best to propagate $\kappa$ ``immediately'', that is the time iteration after the time of the changed $\kappa$.
Initially the propataion of the activity level waited until the first spike of the node. This gave unrealistic time delays in the network. %TODO Finn/Lag figur! Skriv om dette tidligere!
% TODO TODO TODO TODO TODO TODO TODO TODO TODO TODO TODO TODO TODO TODO TODO TODO TODO TODO TODO TODO TODO TODO TODO TODO TODO TODO TODO TODO TODO TODO TODO TODO TODO TODO TODO TODO TODO TODO TODO TODO TODO TODO TODO TODO 
% TODO Dårlig under her . Litt rotete over også..
% TODO TODO TODO TODO TODO TODO TODO TODO TODO TODO TODO TODO TODO TODO TODO TODO TODO TODO TODO TODO TODO TODO TODO TODO TODO TODO TODO TODO TODO TODO TODO TODO TODO TODO TODO TODO TODO TODO TODO TODO TODO TODO TODO TODO 
%It was found that the time delay introduced by waiting for the first spike gave a delay thought  to be to large. In this implementation the activity of the node propagates as soon as the input value changes.
%
%The $\kappa$N could also give output as a function of the only its input. 
% TODO TODO Få med at neste setning bare gjelder "in respect to the output .."
If the node propagates its signal ``immediately'', it could be seen as stateless with respect to its output. %, and becomes a variant of the second generation ANN for this aspect of the output. %XXX sjå over setn. Spess siste 5 ord.
The output then becomes a variant of the output given by a node of the second generation ANN.
% 		% 		%
The activation function is based on a mechanistic model of the neuron, and differs somewhat to that of a node of a second generation ANN.
%If we instead let each $\kappa$N give an output as a function of the present and estimated future input, and give a similar output, the network becomes something else.
%The output now varies as a function of the input, and the node becomes could be made stateless in this respect. 
%In this case the ANN nodes become very similar to a node of a second generation ANN.
As output of a node is given by the activity level of the node, not the state of the node, we have that a $\kappa$ANN is able to communicate with an ANN of the second generation.
% XXX Få også med at i tillegg har den mulighet for å gi anna type output. Til SANN, som spikes..

We also have the oppurtunity to calculate the ``depolarization'' value of the nodes by the concept of ``time windows''. 
%We still have the oppurtunity to calculate the value of the nodes by the concept of ``time windows''.
% TODO Skriv at vi kan kalkulere spike times.
By equations developed for this project, the $\kappa$N also has the ability to calculate the time of its next spike.
By making specialized synapses for this purpose, a $\kappa$N could therefore communicate with a node of the third generation ANN, SANN.
%By making specialized synapses for this purpose, a $\kappa$N could also communicate with a node of the third generation ANN, SANN.
To achieve this, a special synapse based on the SANN synapse has to be developed for the conjunction of a KN and a SN.
%To achieve this, a special synapse based on the SANN synapse has to be developed for the connection between a KN and SN.
This synapse should then be based on the synapse used in the SANN model, where only the spike is transmitted.
%If this synapse is created by the model used for the SANN synapse, where only the spike is transmitted, nodes of the $\kappa$ANN model is able to communicate with a SANN node.  %makes it possible 

%If desired, the spike time and state of the neuron can still be calculated. 


An other aspect that appeared as a consequence of the development of the $\kappa$ANN was the concept of time scheduling as done in $\kappa$ANN.
In this concept, each node contains a variable giving its planned task time.
When the time arrives, a pointer to the element is inserted into \emph{pWorkTaskQue}.
%TODO Skriv vidare her..
% TODO TODO TODO TODO TODO TODO TODO TODO TODO TODO TODO TODO TODO TODO TODO TODO TODO TODO TODO TODO TODO TODO TODO TODO TODO TODO TODO TODO TODO TODO TODO TODO TODO 
%Fortsett litt, her. Tenker at det neste kan stå sist..




In this implementation, it is possible that the full advantage of the new model is neglected in the desire to compare the model to the SANN model.
Because this was first discovered when the results were analyzed, and due to little time, this will be the first continuation of this project.			%this will be placed under "for furter work".
An other aspect that was not tested was the efficiancy of the new model.
The inital desire to develop this ANN model was that simulation of a time--variant variable used to much computational resouces.

Another aspect is that for a SANN node, a large amount of work is needed to maintain the activation level of a node, even if the activation level of the node is constant.
%For a node that fires at a constant rate, the calculation required to maintain this firing rate would be constant.
For a situation where a node recieves input that makes if fire at a constant, high rate, the calculations required by the SANN model would give a high, constant work load.
For a node of the $\kappa$ANN model, the same situation would initially require much computational resouces.
After the node's responce to this activation level was calculated, the computations required to maintain this level would go to almost zero.
%The same situation would for the $\kappa$ANN model would initially require much computational resouces, but after the node's responce to this activation level was calculated, the computations required would soon go to zero.
%FORTSETT HER!

%så gå over til sammenligning av denne mot SANN.



\section{Discussion 2. Denne skal nok vekk.}

% TODO TODO TODO TODO TODO TODO TODO TODO TODO TODO TODO TODO TODO TODO TODO TODO TODO TODO TODO TODO TODO TODO TODO TODO TODO TODO TODO TODO TODO TODO TODO TODO TODO 
% DETTE er ikkje diskussjon. Dette er implementasjon eller noke.      TODO FLYTT TIL design-kapittel. Siste section: discussion (eller noke)
% TODO TODO TODO TODO TODO TODO TODO TODO TODO TODO TODO TODO TODO TODO TODO TODO TODO TODO TODO TODO TODO TODO TODO TODO TODO TODO TODO TODO TODO TODO TODO TODO TODO 

%En eller anna plass: (ikkje akkurat her) Skriver her fordi eg har inspirasjon no, og ikkje vil leite..) 	Ja! Kanskje i "conclutions"XXX
Implementing the mechanisms of a neural simulator is not trivial, even without optimizing it for run time efficiency.
%todo SKriv om neste linje! xxx
Even if this is not an important aspect of this project I have tried to, wherever possible, optimize the implementation for efficiency.
%Skrive kvifor: Om at designet er optimalisert både for generalitet (for å gjøre utvidelser/endring lettere) og for effektivitet. DETTE for å kunne bruke implementationen videre (personlig, eller for andre).

The functions that are most often called are inlined. This means that I have given a hint to the compiler to put the compiled code wherever the functions are called.
This causes the function calls to run faster but also increases the size of the executable file, so this should be used with caution. %Kanskje skrive dette, men også da skrive størrelse på endelig program. ca. 0.5 M (?)
For this implementation, size will not be a problem. %Kvifor?
Inlining of functions are still kept to a minimum, in case of further work on this software. %skriv annaleis. Kvifor "in case of further work on this software."? Forklar bedre, eller skriv om (anna argumentering).

Also in other parts of the implementation, the code is written with a focus on possible future expantion.
%Difor er ting laga enkel å forstå for en som kan nevro: oppsettet av nodene er lagt opp som det biologiske neuronet.
The object model is designed to be general for the two models, both to make the two implementations more comparable for this project and to make the implementation better suitable for future comparison.

The design of each node is based on the biological neuron to be more intuitive for programmers with knowledge of neuroscience. 
This is not only to make the implementation easier to use for potential future programmers, but also because little is known about what is important in neuroscience.
If the implementation is constructed strongly inspired by the emulated system, with multible elements constructed in the same way as the original system, expantion and modification of the indivitual elements involve less effort.
Say, for example, that new aspects are discovered tomorrow. In this case, the code can easier be modified after this discovery.

This principle does not only account for future uses by other programmers.
In multiple occations, aspects that are where new to me have been implemented at after the main functionality of the classes where designed. 
This required less work because I followed principles that was important to Bjarne Stroustrup during the creation of C++; To make the design general and open and suited for any future uses.%ELLER NOKE Siter"TheDesignAndEvolution of C++".


% TODO TODO TODO TODO TODO TODO TODO TODO TODO TODO TODO TODO TODO TODO TODO TODO TODO TODO TODO TODO TODO TODO TODO TODO TODO TODO TODO TODO TODO TODO TODO TODO TODO 
% dette ER diskurs (?) :
One element that is not implemented, is axo--somatic and axo--axonic synases. 
This is synapses where the input to the postsynaptic neuron enters at other places of the neuron than the dendrite. 
In biology, most inhibitory synases have theire input close to the soma of the postsynaptic neuron. 
This will give the inhibitory input less delay compared to the excitatory input, and might be an important aspect in neural computations.
This can be implemented easily if this is found to be important for some future use of this code.

An other important aspect that is not impelented is axo-axonic synapses, linked to short term synaptic plasticity.
%XXX An axo--axonic synapse is a synase that enters the postsynaptic neuron somewhere close to one of its output synapses.
% Gjør om rekkefølgen litt. Skriv om short-term syn.p. først, så evt. axo-axonic synapses.
Short--term synaptic plasticity is synaptic plasticity that does not have any long term effect of the neuroal network (learning), only on more immedate aspects of signal transmission.
In particular, we have the synapses that gives input near the axon terminal of the neuron (axo-axonic synapses). This will alter the depolarization at the neurons output synapses. %og auke/minke mengde Ca2+ i presyn. bit av synapsen.
This will not cause any transmission to occur, only ``prime'' the presynaptic membrane of the synapse for transmission.
When the next action potential arrives, the size of the transmission will be larger than usual (or less, depending on whether the axo--axonic synapse is excitatory or inhibitory).
Again, neither short--term synaptic plasticity or axo--axonic synapses has been implemented because this was not an aspect of this project and due to time constraints.
% Skriv heller at short--term syn.p. ikkje er implementert, så axo-axonic synapses ville ikkje auka funksjonaliteten til simulatoren.
Implementing it will involve less work due to the implementation design in this code.
% Denne setninga er for å vise at det er lett å innføre. Skriv om.

If the goal of the implementation is to simulate a neural network from biology, spatial and temporal resolution is more important than efficiency of the calculation.
%When it comes to temporal and spatial resolution, this can easily be extended.
In this implementation, spatial and temporal resolution can easity be extended.
If we want to increase the accuracy, this can be done by increasing the number of elements in each node (and making the timestep accordingly smaller).
This will also make the computational efficiency of the simulator less, and the pragmatic use of the simulator will suffer.

If we need a better accuracy, and for example double the number of serial elements in each node and halve the size of the timestep, the functioning of each element will be the same.
The temporal and spatial resolution will be increased two-fold at the expence of the computational efficiency of the simulator.

With ``spatial resolution'' i refer to the ability to separate between elements located at different positions in space.
This is important for the output, scince different output synapses are situated at different locations of the neurons axon.
This will cause different transmission delays, and might be important for the neural calculations.
% Also the axon will be have more elements, and different synapses along the axon may have differenti time delay.
This gives us the ability to make a separation between ``early synapses'' and ``late synapses'' along the axon, and gives us a better spatial resolution for the simulated neuron.

The whole reason for developing the third generation ANN, spiking artificial neural networks, was the growing focus on the timing of the different events within the neural network.
This is computational demanding, and high resolution simulations is not suited for real time pragmatic uses of ANNs. For simulations used to test hypothesis in neural science, however, accuracy is most important.
The focus on generality in this implementation will therefore make the code reusable for possible future pragmatic uses and for implementing neural simulators.

If the new model is more effective than the old model for spiking neural networks, we can not know whether this is only goes for one of these uses. 
For this reason I found it best to implement as generally as possible, for future efficiancy comparison between the two models.
%Skrive eksempel: "More specifically, if we divide the axon into smaller pieces, spatial accuracy is better."
%Skriv om forskjellane mellom KANN og SANN. Her kommer kanskje største fordelen med KANN? (Kan kanskje legge inn vilkårlig antall element, til tilsvarende størrelse på 'time step').

%slutt: En eller anna plass: ....














\section{Ting som kanskje er nye}
Skriv litt om optimaliseringa gjordt i SANN. Simulert asynkron tid istedetfor oppdatering kvart tidssteg.







\section{Kva burde eg gjordt annerledes?}
Skriv at fokuset mitt i denne implementeringa var å sammenligne de to modellene. Dette gjorde at eg satt opp auronet på eg spesiell, og ikkje-optimal måte.
(Både for effektivitet, men også for implementasjon. Det var kanskje vanskeligere å implementere modellen på måten eg gjorde det, enn nødvendig. Kann trenger bare eit auron, med synapser ut.)
Dette burde eg gjordt annerledes, og både implementering og effektivitet ville vunnet på dette.




Burde ikkje bygd de to implementasjonane så lik. (?)
Det er vanskelig å sammenligne de to. For K\_auron kunne eg ivertfall droppa dendrite og kanskje axon.
Jeje - vettafaen eg.

Burde ikkje brukt så mykje tid på pEstimatedTaskTime-lista. Kanskje eg kan argumentere for at dette kan være nyttig, men for dette prosjektet er det tidssløs.

\section{Feiler og mangler}
Skriv om feiler og mangler, forenklinger, ting eg ser bort ifra, ting eg går ut fra, osv.

%Forenklinger:
%	- Synapsen ser bort fra tidsfenomener (f.eks. potentiation). Skriv at dette er det få som har med.
%
% Skriv også om ting er har med, som ofte er forenkla vekk:
% 	- Tidsdelay i synapse, dendrite, axon.
% 	- Mulighet for å utvide axon og dendrite dersom eg ønsker bedre spatial resolution.
% 	- samme for temporal resolution.
% 	- Implementert det-derre-prinsippet (Dale 's(?)) . Neir ikkje heilt. Har ikkje gjort dette for heile neuron, men enkeltsynapsene er enten E eller I. Dette er vanlig å ikkje ha med.
% 	- I fremtidig arbeid: Tenker å lage ulike synaptisk plastisitetsregler. Muligheten for dette er open..
% 	-
%

\section{Kva gjenstår / TODO / For further research}
Eit element eg ønsker sterkt å gjennomføre er å sammenligne effektiviteten til den nye modellen til andre modeller. Eg føler modellen har store muligheter for effektivitet, spesiellt for neuralkretser med der nodenes $\kappa$ er konstant.
Dette vil ikkje innebære nevneverdig computational load for nettet.

Synaptisk plasticitet. Ønsker å see på synaptisk plastisitet for neuralnettet.
Ønsker også å utvikle nye modeller for dette. (basert på K(1-exp(-at) ligning)
Tilsvarende for refraction period..

Transferfunk. ANN?

%\section{Effektivitetsanalyse: Korleis gjennomføre}
%Kan lage ei while-løkke, som lager eit nytt neuron i kvar iter, og lager synapse til eit E neuron.
%
%Slik kan eg lage eit testneuron med f.eks. 1000 input (kan referere til at dette er estimerte vanlige nivået).
%Kan dermed sammenligne SN og KN med hensyn på effektivitet.
%
%Dersom eg vidare lar sensor funk variere kvar tiande tidssteg for ti prosent, 20. tidssteg for 20 % osv.
%Så har eg noke å skrive om i forhold til testoppsett.
%









% XXX Skriv at resetting til v_r etter AP er ikkje instantaneous. Dette tar også litt tid. For videre arbeid vil også dette bli implementert!

% XXX Skriv at KANN er en mellomting mellom fANN og SANN. Det har muligheten til å kommunisere med begge.
% 		- fANN kan overføre aktivitetsnivået sitt direkte til KANN (Kan sette $\kappa$ = input-fra-fANN) -- Begge veier (KANN: kan få info FRA, og gi info TIL fANN).
% 		- SANN kan overføre aktivitetsnivået sitt indirekte til KANN (KN kann analysere input til $\kappa$. KN kan gi output til SN (direkte))
% Det kan være dette er eit stort bidrag for ANN-verden. Dersom det i tillegg er meir effektivt, så ...


%FRA KAPITTEL: \chapter{resultat av sammenligning: SANN vs. $\kappa$ANN}

%Konklusjon: 	Skrive at implementasjon er noe vanskeligere for KANN, da det også har med fremtidsutsikter.
% 				Dette kan forventes, siden KANN noder kan sees på som implemenasjon av en Mealy automata og SANN en Moore automata av spiking neuron.

\section{ Mulige aspekt som er nye i denne oppgava }
	I likhet med da eg først utvikla SANN, trudde eg at tidsmodellen min var heilt ny. Det at nodene ikkje ble oppdatert kvar iterasjon, men bare ved "events". Dette er feil. "event--driven simulation of SANN".
	
	Er rimelig sikker på at KANN--modellen er heilt ny. Har ikkje hørt om denne, har ikkje funnet den, alle de professorene eg har snakket med har ikkje hørt om slikt, osv.
	
	Kanskje: Leste at det var vanskelig å estimere fyringstid. Dette blir isåfall løst ved KANN.

	Trur KANN gir muligheten for 'abituary time steps' uten å bruke meir prosessorkraft. I så fall kan dette være stort! Kan gi større oppløsning i forhold til 'spatial resolution' også.
	Fordi KANN beregner bare ved endring av node input, ikkje for alle tidssteg..
	(Dette bør også simuleres, slik at eg får data til å støtte meg på)  	Fy faen, dette er fett isåfall!


	%ikkje akkurat denne overskrifta, men på en eller anna måte vil eg diskutere kva som er nytt i denne oppgaven
	\subsection{Possible new elements from this project}
	Despite an extensive extensive litterature search an asking multiple proffessors on the subject, I have been unable to find ANN coding the activity of the nodes as I have done in this project.
	In fact, everything I have found indicates that the most used activity variable used both in ANN and in neuroscience is the frequency of the neurons.
	
	It is, in other words, entirely possible that the mathemathics developed for this project is entirely new. 
	In 	this case, I believe that this could be an important contribution to the field of neural networks (espescially the study of biological neural networks).

	I am no expert on neural systems, but in the course of testing the new method for making ANN I have found phasic firing of a node as a consequence of a sinusiodal input %SJÅ datafiles_for_evaluation/filer_til_rapport/
	, witch is an known phenomoen in neuroscience called Local Field Potential Oscillations (LFPOs).
	
	BLA BLA BLA..


	\section{ Konklusjon }


%	\subsubsection{Propagation of $\kappa$}
%	%Skriv at det er vanskelig å vite når kappa skal propagere. Skriv litt rundt dette: kva som er intuitivt (ved fyring), beste er å ha transient overføringsfunk men dette krever transferfunk. fokus, 
%	%umulig å vite kva som gir godt nok resultat, så bør gjøre dette eksperimentellt. : sammenligne depol.-forløp, og sjå når de ikkje lenger er lik. Kanskje etter en ekstra periode, lenger tid, anna? 
%	% 	Skal ivertfall vente ei stund. Resultatet om KANN er meir effektivt er avhengig av svaret på dette spm.
%	% KANSKJE DETTE IKKJE SKAL STÅ HER? VEIT IKKKJE KOR..
%	%ELLER: kan skrive resultatet her, etter at eg har sammenligna resultata..
%	For $\kappa$ANN it is hard to know when the activity variable, $\kappa$ should propagate. In many ways, $\kappa$ gives the general input of a node.
%	$\kappa_{ij}$, node $j$\,s influence on node $i$ is given by the period of node $j$, given by $\kappa_j$. 
%	It is wery hard to deside some specific time instance $\kappa_j$ should propagate, since the period when $\kappa_j$ is changed also influences $\kappa_i$. 
%	The period of neuron $j$ and thus the period between synaptic transmissions to neuron $i$ can also be said to be given by the new $\kappa_j$ first after the first period with the new $\kappa_j$.
%	
%	The best is to make a transfer function from $\kappa_j$ to $\kappa_i$ with a transient time course. 
%	Due to time limitations, this will unfortunately be outside the scope of this project, since this will further alter the postsynaptic nodes after node $j$. 
%	A neural network of transfer functions, where each node represents a transform, will probably give a better result both in terms of efficiancy and in terms of the transient time course of each node.
%	
%	What can be established from the above discussion is that the time of propagation should be somewhere between zero and two periods after updating $\kappa$, depending on when $\kappa$ is updated in the period.
%	Because of the caotic and unpredictable nature of neural networks, and the simplification to deside some constant time delay, the time delay between updating $\kappa$ and propagation of $\kappa$ have to be found experimentally.
%	As the effectivity of $\kappa$ANN is strongly dependent on this aspect, it is wery hard to theoretically find the efficiancy of the new model.
%
%	To experimentally find when $\kappa$ propagates, the transient time course of each neuron have to be compared.
%	For pragmatic ANNs, it is also hard to find a measure of ``good enough'' for the comparison of the two plots. 
%	SKRIV MEIR NÅR RESULTAT DUKKER OPP! %TODO
%	%TODO Når eg har sammenligna de to, skriv meir her. Bra nok kan være "rett" output for ut--noden, eller "rett" transient time course for ut-noden, osv. Sammenligner heile tida med den andre modellen som om den var rett.
%	% Dette kan også være feil. Kven veit.. Mykje å skrive på diskursjon etter dette aspektet!









