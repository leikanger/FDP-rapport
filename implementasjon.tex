
\chapter{ Implementation  -- Denne kan kanskje takast vekk}
% Her skal eg skrive om implementasjon av det som er planlagt i "Design".
% Kvifor C++ ?


	%Innledning, kva programmeringsspråk brukes, osv. Etabler eit utgangdspunkt for resten av teksten!
	Når man implementerer for å sammenligne to modeller er det best at de to er mest mulig lik. Dette kan lett gjøres vha. arv. Peiker mot OO-språk.
	
	Det sterke fokus på være mest mulig likt biologiske neurale system peiker mot OO-språk virker bra. Da kan man dele opp neuronet i "compartments", som i utgangspunktet er adskilt.

	C er eit språk som er effektivt (raskt), samtidig som det har vore undervist på kyb.

	= C++.

	Etterkvart: Skriv litt om Stroustrup's anbefalinger om å bruke stl.




\section{Fokus}
I denne oppgaven er ikkje fokus på effektivitet av utregningene, men evt. effektivitetsdifferanse mellom de to. 
Først brukte eg integer for å beskrive f.eks. depol. og Kappa, men dette førte til en del avrundingsfeil. 
Gjekk dermed over til å bruke flyttal for å beskrive aktivitetsvariabel. (denne notaten er notert før eg har endra de fra int til float.. Sjå korleis det går.



\section{Notater:}
\subsection{Skruve om std::vector vs. list.}
Eg har brukt en halv dag på å teste om eg skal bruke vector eller list for pAllAurons og pAllKappaAurons. Konskluderte med at det ikkje hadde noko å seie.

Hadde 101 test-auron (K\_auron) og eit sensor-auron. De var ikkje kobla ihop, og hadde kvar en Kappa på 2.07*FYRINGSTERSKEL. Sensorauronet hadde en sinus-varierende kappa.
Kjørte 10.000 tidsiterasjoner. Resultat av kjøretid:

Vector variant:
15.626 15,537 14,8 13,3 13,5 13,6

List variant:
13,46 13,41 	13,72 13,6 14,9 14,7 13,3 14,9

Konkluderer med at de ikkje har stor nok forskjell til å bry seg.

(dette kan være smart å skrive inn i rapporten. (og da blir ikkje denne halve dagen fullstendig bortkasta..))

