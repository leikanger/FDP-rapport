\section{resultat av sammenligning: SANN vs. $\kappa$ANN}
	\section{ The transient course of the auron's depolarization }
	Her skal plott av depolarization legges.

	\section{ Det eg får tida til å sammenligne - implementasjon, effektivitet, ...}
	Skrive at implementasjon er noe vanskeligere for KANN, da det også har med fremtidsutsikter. Dette kan forventes, siden KANN noder kan sees på som implemenasjon av en Mealy automata og SANN en Moore automata av spiking neuron.

	\section{ Mulige aspekt som er nye i denne oppgava }
	I likhet med da eg først utvikla SANN, trudde eg at tidsmodellen min var heilt ny. Det at nodene ikkje ble oppdatert kvar iterasjon, men bare ved "events". Dette er feil. "event--driven simulation of SANN".
	
	Er rimelig sikker på at KANN--modellen er heilt ny. Har ikkje hørt om denne, har ikkje funnet den, alle de professorene eg har snakket med har ikkje hørt om slikt, osv.
	
	Kanskje: Leste at det var vanskelig å estimere fyringstid. Dette blir isåfall løst ved KANN.

	Trur KANN gir muligheten for 'abituary time steps' uten å bruke meir prosessorkraft. I så fall kan dette være stort! Kan gi større oppløsning i forhold til 'spatial resolution' også.
	Fordi KANN beregner bare ved endring av node input, ikkje for alle tidssteg..
	(Dette bør også simuleres, slik at eg får data til å støtte meg på)  	Fy faen, dette er fett isåfall!

	\section{ Konklusjon }
